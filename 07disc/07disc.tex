\chapter{Discussion and Future Perspectives}
\label{07disc}
\colorbox{yellow}{STLL WIP}

\section{Building Accessible and Automated Tools for MC Data Analysis}

In this work I have shown CyGNAL's capabilities, describing in detail its design and inner mechanisms, and outlining its usefulness with regards to the analysis of \acrshort{mc} datasets.

The main testament for the usefulness of the tool is the fact that it has become a part of routine \acrshort{mc} analyses in our lab. With its support for plain text to FCS inter-compatibility, one can seamlessly integrate with \acrshort{mc} platforms such as Cytobank. Additionally, as the user only need to run simple Python commands on the terminal to use CyGNAL, it has been readily adopted in day-to-day lab use even by users with no advanced computing experience. 
As I have shown in Chapters \ref{02methods} and \ref{03cytof}, CyGNAL is able to perform a comprehensive analysis of changes occurring across the samples of the often wide \acrshort{mc} experimental systems. Designed with a studying \acrshort{ptm} signalling changes in particular, CyGNAL's computation of \acrshort{emd} and \acrshort{dremi} scores resolves marker intensity and connectivity changes. The easy to use and customisable interactive Shiny-Apps allow for both exploratory and close to publication-grade visualisation of the results. Tools are meant to be used, and that publications by colleagues such as Michelozzi \textit{et al}.\cite{michelozzi_activation_2023} employed CyGNAL is a testament to its relevance.

Originally meant as a simple exercise in curiosity-driven exploration after noticing the correlation between so called PTM and 'cell-state' markers, and empowered by the tediousness of manually gating the datasets in our lab, the \acrshort{rf} cell-state classifier has become a convenient tool to automate cell-state labelling of \acrshort{mc} datasets in relation to cell-cycle phases.

Built around a simple \acrfull{rf} architecture, the \acrshort{rf} classifier benefits from the fundamental gate-like logic of both decision trees and the manual cell-state gating process. However, I expect the classifier to suffer from generalisation issues when dealing with external data labelled using different workflows. Furthermore, even if it leverages fuzzy logic to match channel names from the model to the input data, the classifier still relies on matching markers found in both the training data and the data to label. While the markers dedicated to apoptosis and cell-cycle phases generally belong to the less variable portions of \acrshort{mc} panel design, this can still pose an inconvenience when deploying the model. However, I have also shown how other weak points such as low performance for apoptotic class prediction using the 5-marker \acrshort{mc} model, can be effectively addressed by just the addition of an additional apoptotic marker to the panel design. Furthermore, the model seems resilient to cell-type composition and even to broad cell-state changes induced by chemotherapy.

Hearkening back to the link between \acrshort{ptm}s and cell-cycle, the 10-marker \acrshort{mc} model also reveals how certain \acrshort{ptm}s prove more informative when training than \emph{bona fide} cell-cycle markers, and discrepancies between expected state and \acrshort{ptm} correlations from the literature and feature importance rankings have anecdotally been used to validate under-performing antibodies with high unspecific background staining.

Both these tools remain relatively under continuous support, and the idea is to eventually merge both code bases and integrate automated cell-state classification into CyGNAL using pre-built classifier models or allowing for the generation of new models based on specific user-provided labelled data.
CyGNAL could also be augmented by the addition of PHATE~\cite{moon_visualizing_2019} as an alternative dimensionality reduction step, and implementing a third type of Shiny-App to visualise these embeddings and overlay user-selected metadata labels or antibody intensities.


\section{Charting Stromal and Oncogenic Regulation of CSC Polarisation}

Single-cell technologies can describe cell-cell communications and cell-type transitions in complex organoid settings and \emph{in vivo} tissues~\cite{qin_deciphering_2020, sqjin_sqjincellchat_2021, bues_deterministic_2022}. As shown in Qin \emph{et al.}~\cite{qin_cell-type-specific_2020}, a heterocellular colonic epithelia organoid system can be employed in experimental designs covering the effects of both intrinsic CRC oncogenic mutations and extrinsic cues. However, the directed and limited nature of the \acrshort{mc} antibody panels used in Qin \emph{et al.}~\cite{qin_cell-type-specific_2020} presented with a limiting factor towards detailed description of colonic organoid epithelial polarisation by intrinsic and extrinsic cues. 

Therefore, in Chapters \ref{04seq} and \ref{05vr} I have employed a multiplexed \acrshort{scrnaseq} analysis of heterocellular CRC organoid cultures to chart a continuous landscape of intrinsic and extrinsic regulation of colonic stem cell states. I have found that stromal cues transition the epithelia towards the \acrshort{revcsc} state, oncogenic signalling pushes the organoid towards \acrshort{procsc}, and exogenous ligands overlapping with both stromal and oncogenic signalling cues can polarise towards both states at once. I have also developed a method to capture these transitional processes, the \acrfull{vr} score, and established a workflow to project it onto Waddington-like data-driven landscapes. 
The work presented in this thesis was paired with complementary \emph{mc} experiments in Qin \& Cardoso Rodriguez \emph{et al.}~\cite{cardoso_rodriguez_single-cell_2023}, where we interrogated colonic stem cell regulation at scale to functionally understand the polarisation mechanisms (Appendix \ref{appendix:preprint}).

Here I have shown how the transcriptomic profiles of epithelial, fibroblast and macrophage cells from the heterocellular cultures can be used to describe inter-type heterogeneity and recapitulate the distinct epithelial compartments. 
The observed \emph{Cd34} high and low fibroblast populations are reminiscent of \emph{in situ} intestinal fibroblast heterogeneity, wherein \emph{Cd34} expressing fibroblast from the bottom of the crypts support the intestinal stem niche whereas \emph{Cd34} low fibroblast are found above the crypt's bottoms and help maintain the BMP gradient needed for epithelial differentiation. While we observed some transcriptional differences between these two fibroblast populations, their regulation of the epithelial compartment remained consistent, possibly due to shared TGF-\textbeta\hspace{0.1cm} between the two. 
Myeloid macrophage transcriptomes formed a continuum trajectory of putative inflammation-related roles, unlike the distinct fibroblast and epithelial populations. However, neither macrophages as a whole nor the extremes of their transcriptional continuum differentially regulated the epithelial cells.
The healthy small intestinal and colonic epithelia is supported by a stem cell niche at the bottom of the crypts regulated by both intrinsic and stroma-secreted signalling gradients. These traditional \acrfull{csc}s however, are not the sole stem cell state, with less common low-proliferative \acrfull{revcsc}s being able to replenish the \acrshort{csc} niche and repair the epithelial tissue in response to tissue damage~\cite{ayyaz_single-cell_2019}. Here I have shown how these \acrshort{revcsc} are enriched by stromal WNT and TGF-\textbeta\hspace{0.1cm} when WT organoids are co-cultured with fibroblasts, and how \acrshort{revcsc} also resemble public descriptions of the same population and a "foetal"-like state~\cite{mustata_identification_2013}.
The gradient of organoids with accumulating oncogenic mutations revealed how a \acrfull{procsc} state is enriched in CRC organoids. These cells are present in lower numbers in WT and \textit{shApc} organoids, but quickly dominate the landscape of stunted absorptive and secretory differentiation in the \textit{shApc} and \textit{Kras\textsuperscript{G12D/+}} (AK), and \textit{shApc}, \textit{Kras\textsuperscript{G12D/+}} and \textit{Trp53\textsuperscript{R172H/–}} (AKP) colonic organoids. \acrshort{procsc} were found to be transcriptionally similar to other cells from mouse models and human CRC.

With a clear differential regulation by extrinsic stromal cues and intrinsic oncogenic signalling, polarisation of WT colonic epithelia towards both \acrshort{procsc} and \acrshort{revcsc} could nonetheless be achieved via exogenous WENR added to the culture media. These findings, together with subsequent \acrshort{mc} validation~\cite{cardoso_rodriguez_single-cell_2023} of signalling hubs identified via cell-cell communication analysis, suggest that both states are part of a shared polarisation landscape with overlapping signalling hubs that overlap and compete with one-another to establish colonic epithelial cell-fate.
In this context, the observed breakdown of fibroblast-to epithelia communications in CRC organoids (at least partly due to downregulation of key signalling receptors by the epithelial cells) seems to suggest that intrinsic oncogenic cues dominate extrinsic stromal cues. The interplay between the two with regard to \acrshort{procsc} and \acrshort{revcsc} polarisation is explored further in Qin \& Cardoso Rodriguez \emph{et al.}~\cite{cardoso_rodriguez_single-cell_2023}, where we established that TGF-\textbeta\hspace{0.1cm} can induce \acrshort{revcsc}-like cells in CRC organoids in the context of low PI3K signalling, supporting the suggested role of \acrshort{revcsc} as a drug-resistant state in CRC that can drive relapse after chemotherapy~\cite{alvarez-varela_mex3a_2022, zapatero_trellis_2023}.

\emph{In silico} analysis of cellular dynamics has suggested that the epithelia transitions towards \acrshort{revcsc} following a polarisation processes from adjacent cell-sates, while \acrshort{procsc} dominance of the epithelia is achieved thanks to its proliferative potential.

I postulate that cellular pluripotency scores and rates of transcriptomic change capture the cellular dynamics of systems such as the colon epithelia, providing for an avenue towards generation of data-driven Waddington-like landscapes of cellular differentiation and plasticity. The \acrfull{vr} score described in Chapter \ref{05vr} synthesizes both CCAT and RNA velocity vector length metrics to capture coarse pluripotency changes and global transcriptomic structure thanks to PHATE. Finer details at a local level capture the availability of cell-sates on as determined by RNA velocity. Reconstruction of landscapes with ynthesis of processes as data-driven lasncapes representation. Powered by VR score, capture coarse pluripotency and transcriptomic strcuture. Local structure captures state avaialabiliyt as determined by RNA velocity.
Reconstructed landscapes recreated shared landscapes of polarisationand local transcriptomic. revcsc as an accesible state with stromal ligands, with oncocenic mtuations trapping the organoid in procsc.
    % When applied to murine organoid perturbation system described in Chapter \ref{04seq}, the VR landscapes depict a picture of a shared differentiation that can be traversed through cell-extrinsic ligands or cell-intrinsic oncogenic mutations. In particular, the increased availability of \acrshort{revcsc} in the presence of stromal ligands (Figure \ref{fig:4da}) can also be observed on the VR landscapes (Figure \ref{fig:5land}B). Furthermore, the collapse of stromal-to-epithelial communication in cancer organoids (Figure \ref{fig:4cc}A) and their lack of \acrshort{revcsc} polarisation (Figure \ref{fig:4da}C) is reflected in the tarn-like topology of the AK VR landscapes, where the bulk of the organoid appears trapped in the \acrshort{procsc} state.

Put together, these results describe fibros as master stromal regulators and CRC as a hyperproliferative trap. Given limitations of all organoid work and no insitu data, with no depe exploration of non paracrine stromal interactions or role of cafs. Further understanding triggering/blocking the rev and pro states necessary, specially as revcsc appear as a putative target to tackle emergence to chemoterapy resistance by blocking plasticity processes towards it. In terms of implemntation, increase accsibiliyt of VR landscapes by distributing it as an nbdev project, just like the pykrack package (Appendix \ref{appendix:pykrack}), but with interactive renderings of the generated landscapes. 


Last part discussing again relevance, revcsc as a baddie to target in cancer, and further understanding on triggering/blocking the rev and pro states necessary,specially regarding cafs, OF course also mention limitation of all organoid work, non paracrine effects, and with no in situ animal or human validation.
    % Our work and others now collectively suggest that fibroblasts are master regulators of revival stem cells in both the small intestine and colon.
    % Rather than establishing an entirely new cancer-specific cell-fate, our study suggests that oncogenic mutations cell-intrinsically polarise cells to an extreme yet pre-existing proCSC state, while simultaneously disrupting cell-extrinsic regulation of plasticity -- trapping cells as proCSC. These results describe cancer as a chronic, unidirectional shift in de-differentiation.
    % Collectively, our results and others suggest fibroblast-induced revCSCs may represent an important 'drug-tolerant persister' (DTP) state in CRC. Given that targeting cell-plasticity is an emerging area of cancer therapies \cite{burkhardt_mapping_2022}, future studies could target CRC DTP cells by combining YAP inhibitors (to block access to DTP revCSC) with standard chemotherapies (to kill proCSC). Helped by better understanding of caf effect Future cell-cell communication studies between CAF sub-types \cite{sahai_framework_2020} and defined epithelial genotypes could uncover exceptions to the signalling models described here and therefore provide novel avenues for therapeutic intervention in CRC.


\section{Knowledge Graph}

Coupled with emerging multimodal approaches that can capture at once intra and intercellular communication features, like phospho-seq \cite{blair_phospho-seq_2023} or Jamie's signal-seq. -> perfect tool to solve inter- and intra-cell communications at the single cell level without relying on clusters and in a modality agnostic way. 

Mixed results tbh.. Too similar to GEx and not good inverse correlation with cc probabilities (woudl expect cluster pairs highly interacting to have lowered distances [thus negative Pearson correlations]). Rsults sugest that diffusion step with wavelets is not enough, supported by work on prunned wavelt banks a larger scales and by projection on omnipath KG, which results in same GEx-like profile.

Fine balance between signal loss determined by lmited number of gene nodes on graph, but also on graph structure itself being important enough to produce results signficantly different from GEx matrix.


