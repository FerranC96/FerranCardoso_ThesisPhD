% UCL Thesis LaTeX Template
%  (c) Ian Kirker, 2014
% 
% This is a template/skeleton for PhD/MPhil/MRes theses.
%
% It uses a rather split-up file structure because this tends to
%  work well for large, complex documents.
% We suggest using one file per chapter, but you may wish to use more
%  or fewer separate files than that.
% We've also separated out various bits of configuration into their
%  own files, to keep everything neat.
% Note that the \input command just streams in whatever file you give
%  it, while the \include command adds a page break, and does some
%  extra organisation to make compilation faster. Note that you can't
%  use \include inside an \include-d file.
% We suggest using \input for settings and configuration files that
%  you always want to use, and \include for each section of content.
% If you do that, it also means you can use the \includeonly statement
%  to only compile up the section you're currently interested in.
% You might also want to put figures into their own files to be \input.

% For more information on \input and \include, see:
%  http://tex.stackexchange.com/questions/246/when-should-i-use-input-vs-include


% Formatting and binding rules for theses are here: 
%  https://www.ucl.ac.uk/students/exams-and-assessments/research-assessments/format-bind-and-submit-your-thesis-general-guidance

% This package goes first and foremost, because it checks all 
%  your syntax for mistakes and some old-fashioned LaTeX commands.
% Note that normally you should load your documentclass before 
%  packages, because some packages change behaviour based on
%  your document settings.
% Also, for those confused by the RequirePackage here vs usepackage
%  elsewhere, usepackage cannot be used before the documentclass
%  command, while RequirePackage can. That's the only functional
%  difference as far as I'm aware.
\RequirePackage[l2tabu, orthodox]{nag}


% ------ Main document class specification ------
% The draft option here prevents images being inserted,
%  and adds chunky black bars to boxes that are exceeding 
%  the page width (to show that they are).
% The oneside option can optionally be replaced by twoside if
%  you intend to print double-sided. Note that this is
%  *specifically permitted* by the UCL thesis formatting
%  guidelines.
%
% Valid options in terms of type are:
%  phd
%  mres
%  mphil
%\documentclass[12pt,phd,draft,a4paper,oneside]{ucl_thesis}
\documentclass[12pt,phd,a4paper,oneside]{ucl_thesis}


% Package configuration:
%  LaTeX uses "packages" to add extra commands and features.
%  There are quite a few useful ones, so we've put them in a 
%   separate file.
% -------- Packages --------

% This package just gives you a quick way to dump in some sample text.
% You can remove it -- it's just here for the examples.
\usepackage{blindtext}

% This package means empty pages (pages with no text) won't get stuff
%  like chapter names at the top of the page. It's mostly cosmetic.
\usepackage{emptypage}

% The graphicx package adds the \includegraphics command,
%  which is your basic command for adding a picture.
\usepackage{graphicx}

% The float package improves LaTeX's handling of floats,
%  and also adds the option to *force* LaTeX to put the float
%  HERE, with the [H] option to the float environment.
\usepackage{float}

% The amsmath package enhances the various ways of including
%  maths, including adding the align environment for aligned
%  equations.
\usepackage{amsmath}

% Use these two packages together -- they define symbols
%  for e.g. units that you can use in both text and math mode.
\usepackage{gensymb}
\usepackage{textcomp}
% You may also want the units package for making little
%  fractions for unit specifications.
%\usepackage{units}


% The setspace package lets you use 1.5-sized or double line spacing.
\usepackage{setspace}
\setstretch{1.5}

% That just does body text -- if you want to expand *everything*,
%  including footnotes and tables, use this instead:
%\renewcommand{\baselinestretch}{1.5}


% PGFPlots is either a really clunky or really good way to add graphs
%  into your document, depending on your point of view.
% There's waaaaay too much information on using this to cover here,
%  so, you might want to start here:
%   http://pgfplots.sourceforge.net/
%  or here:
%   http://pgfplots.sourceforge.net/pgfplots.pdf
%\usepackage{pgfplots}
%\pgfplotsset{compat=1.3} % <- this fixed axis labels in the version I was using

% PGFPlotsTable can help you make tables a little more easily than
%  usual in LaTeX.
% If you're going to have to paste data in a lot, I'd suggest using it.
%  You might want to start with the manual, here:
%  http://pgfplots.sourceforge.net/pgfplotstable.pdf
%\usepackage{pgfplotstable}

% These settings are also recommended for using with pgfplotstable.
%\pgfplotstableset{
%	% these columns/<colname>/.style={<options>} things define a style
%	% which applies to <colname> only.
%	empty cells with={--}, % replace empty cells with '--'
%	every head row/.style={before row=\toprule,after row=\midrule},
%	every last row/.style={after row=\bottomrule}
%}


% The mhchem package provides chemistry formula typesetting commands
%  e.g. \ce{H2O}
%\usepackage[version=3]{mhchem}

% And the chemfig package gives a weird command for adding Lewis 
%  diagrams, for e.g. organic molecules
%\usepackage{chemfig}

% The linenumbers command from the lineno package adds line numbers
%  alongside your text that can be useful for discussing edits 
%  in drafts.
% Remove or comment out the command for proper versions.
%\usepackage[modulo]{lineno}
% \linenumbers 


% Alternatively, you can use the ifdraft package to let you add
%  commands that will only be used in draft versions
%\usepackage{ifdraft}

% For example, the following adds a watermark if the draft mode is on.
%\ifdraft{
%  \usepackage{draftwatermark}
%  \SetWatermarkText{\shortstack{\textsc{Draft Mode}\\ \strut \\ \strut \\ \strut}}
%  \SetWatermarkScale{0.5}
%  \SetWatermarkAngle{90}
%}


% The multirow package adds the option to make cells span 
%  rows in tables.
\usepackage{multirow}


% Subfig allows you to create figures within figures, to, for example,
%  make a single figure with 4 individually labeled and referenceable
%  sub-figures.
% It's quite fiddly to use, so check the documentation.
%\usepackage{subfig}

% The natbib package allows book-type citations commonly used in
%  longer works, and less commonly in science articles (IME).
% e.g. (Saucer et al., 1993) rather than [1]
% More details are here: http://merkel.zoneo.net/Latex/natbib.php
%\usepackage{natbib}

% The bibentry package (along with the \nobibliography* command)
%  allows putting full reference lines inline.
%  See: 
%   http://tex.stackexchange.com/questions/2905/how-can-i-list-references-from-bibtex-file-in-line-with-commentary
\usepackage{bibentry} 

% The isorot package allows you to put things sideways 
%  (or indeed, at any angle) on a page.
% This can be useful for wide graphs or other figures.
%\usepackage{isorot}

% The caption package adds more options for caption formatting.
% This set-up makes hanging labels, makes the caption text smaller
%  than the body text, and makes the label bold.
% Highly recommended.
\usepackage[format=hang,font=small,labelfont=bf]{caption}

% If you're getting into defining your own commands, you might want
%  to check out the etoolbox package -- it defines a few commands
%  that can make it easier to make commands robust.
\usepackage{etoolbox}


% Sets up links within your document, for e.g. contents page entries
%  and references, and also PDF metadata.
% You should edit this!
%%
%% This file uses the hyperref package to make your thesis have metadata embedded in the PDF, 
%%  and also adds links to be able to click on references and contents page entries to go to 
%%  the pages.
%%

% Some hacks are necessary to make bibentry and hyperref play nicely.
% See: http://tex.stackexchange.com/questions/65348/clash-between-bibentry-and-hyperref-with-bibstyle-elsart-harv
\usepackage{bibentry}
\makeatletter\let\saved@bibitem\@bibitem\makeatother
\usepackage[pdftex,hidelinks]{hyperref}
\makeatletter\let\@bibitem\saved@bibitem\makeatother
\makeatletter
\AtBeginDocument{
    \hypersetup{
        pdfsubject={Thesis Subject},
        pdfkeywords={Thesis Keywords},
        pdfauthor={Author},
        pdftitle={Title},
    }
}
\makeatother
    


% And then some settings in separate files.
% These settings are from:
%  http://mintaka.sdsu.edu/GF/bibliog/latex/floats.html

% They give LaTeX more options on where to put your figures, and may
%  mean that fewer of your figures end up at the tops of pages far
%  away from the thing they're related to.

% Alters some LaTeX defaults for better treatment of figures:
% See p.105 of "TeX Unbound" for suggested values.
% See pp. 199-200 of Lamport's "LaTeX" book for details.

%   General parameters, for ALL pages:
\renewcommand{\topfraction}{0.9}	% max fraction of floats at top
\renewcommand{\bottomfraction}{0.8}	% max fraction of floats at bottom

%   Parameters for TEXT pages (not float pages):
\setcounter{topnumber}{2}
\setcounter{bottomnumber}{2}
\setcounter{totalnumber}{4}     % 2 may work better
\setcounter{dbltopnumber}{2}    % for 2-column pages
\renewcommand{\dbltopfraction}{0.9}	% fit big float above 2-col. text
\renewcommand{\textfraction}{0.07}	% allow minimal text w. figs

%   Parameters for FLOAT pages (not text pages):
\renewcommand{\floatpagefraction}{0.7}	% require fuller float pages
% N.B.: floatpagefraction MUST be less than topfraction !!
\renewcommand{\dblfloatpagefraction}{0.7}	% require fuller float pages

% remember to use [htp] or [htpb] for placement,
% e.g. 
%  \begin{figure}[htp]
%   ...
%  \end{figure} % For things like figures and tables
\bibliographystyle{unsrt}   % For bibliographies

% These control how many number sections your subsections will take
%    e.g. Section 2.3.1.5.6.3
%  and how many of those will get put into the contents pages.
\setcounter{secnumdepth}{3}
\setcounter{tocdepth}{3}


\begin{document}

\nobibliography*
% ^-- This is a dumb trick that works with the bibentry package to let
%  you put bibliography entries whereever you like.
% I used this to put references to papers a chapter's work was 
%  published in at the end of that chapter.
% For more information, see: http://stefaanlippens.net/bibentry

% If you haven't finished making your full BibTex file yet, you
%  might find this useful -- it'll just replace all your
%  citations with little superscript notes.
% Uncomment to use.
%\renewcommand{\cite}[1]{\emph{\textsuperscript{[#1]}}}

% At last, content! Remember filenames are case-sensitive and 
%  *must not* include spaces.
% I may change the way this is done in a future version, 
%  but given that some people needed it, if you need a different degree title 
%  (e.g. Master of Science, Master in Science, Master of Arts, etc)
%  uncomment the following 3 lines and set as appropriate (this *has* to be before \maketitle)
% \makeatletter
% \renewcommand {\@degree@string} {Master of Things}
% \makeatother

\title{A Thesis Title}
\author{Ferran Cardoso Rodriguez}
\department{Department of Something}

\maketitle
\makedeclaration

\begin{abstract} % 300 word limit
My research is about stuff.

It begins with a study of some stuff, and then some other stuff and things.

There is a 300-word limit on your abstract.
\end{abstract}

\begin{impactstatement}

	UCL theses now have to include an impact statement. \textit{(I think for REF reasons?)} The following text is the description from the guide linked from the formatting and submission website of what that involves. (Link to the guide: {\scriptsize \url{http://www.grad.ucl.ac.uk/essinfo/docs/Impact-Statement-Guidance-Notes-for-Research-Students-and-Supervisors.pdf}})

\begin{quote}
The statement should describe, in no more than 500 words, how the expertise, knowledge, analysis,
discovery or insight presented in your thesis could be put to a beneficial use. Consider benefits both
inside and outside academia and the ways in which these benefits could be brought about.

The benefits inside academia could be to the discipline and future scholarship, research methods or
methodology, the curriculum; they might be within your research area and potentially within other
research areas.

The benefits outside academia could occur to commercial activity, social enterprise, professional
practice, clinical use, public health, public policy design, public service delivery, laws, public
discourse, culture, the quality of the environment or quality of life.

The impact could occur locally, regionally, nationally or internationally, to individuals, communities or
organisations and could be immediate or occur incrementally, in the context of a broader field of
research, over many years, decades or longer.

Impact could be brought about through disseminating outputs (either in scholarly journals or
elsewhere such as specialist or mainstream media), education, public engagement, translational
research, commercial and social enterprise activity, engaging with public policy makers and public
service delivery practitioners, influencing ministers, collaborating with academics and non-academics
etc.

Further information including a searchable list of hundreds of examples of UCL impact outside of
academia please see \url{https://www.ucl.ac.uk/impact/}. For thousands more examples, please see
\url{http://results.ref.ac.uk/Results/SelectUoa}.
\end{quote}
\end{impactstatement}

\begin{acknowledgements}
Acknowledge all the things!
\end{acknowledgements}

\setcounter{tocdepth}{2} 
% Setting this higher means you get contents entries for
%  more minor section headers.

\tableofcontents
\listoffigures
\listoftables


\chapter{Introductory Material}
\label{introlabel}

Some stuff about things.\cite{example-citation} Some more things. 

Inline citation: \bibentry{example-citation}

% This just dumps some pseudolatin in so you can see some text in place.
\blindtext

\chapter{My First Content Chapter}
\label{chapterlabel2}

% This just dumps some pseudolatin in so you can see some text in place.
\blindtext

% \lstset{frameround=fttt,language=Python,numbers=left,breaklines=true}

Test the text. Now with \texttt{cobra.flux\textunderscore analysis}!

% \begin{spacing}{1.5}
% \begin{lstlisting}[caption={Pseudocode snippet for \texttt{1-data\textunderscore preprocess.py}}, breaklines=true,basewidth=6pt,frame=single,language=Python, numbers=left, prebreak=**, postbreak=**, label={lst:code}]
% def run_meteor(model_path, constraints_path, **options):
%   # Load in the model
% 	model = cobra.io.read_sbml_model(model_path)
%   # Load and apply the options
% 	options = options.get('options', None)
% 	model = options_setup.apply_options (model, constraints_path, options)
%   # FBA: Parsimonious FBA
% 	model_solution = meteor_functions.perform_fba(model)
%   # Load in categories for CBA from options
% 	categories = options["categories"]
% 	atpbiomass_reactions = categories['ATP']['biomass']
% 	atpburned_reactions = categories['ATP']['burned']
% 	atpwaste_reactions = categories['ATP']['waste'] 
% 	nadpbiomass_reactions = categories['NADP']['biomass']
% 	nadpwaste_reactions = categories['NADP']['waste'] 
%   # Perform CBA and populate categories
% 	ATP_produced,..., NADP_waste = meteor_functions.cofactor_assessment(
%                                     model_solution, model, atpbiomass_reactions,
%                                     atpburned_reactions, atpwaste_reactions, 
%                                     nadpbiomass_reactions, nadpwaste_reactions)
%   # Populate respective dictionaries for the frontend tables
% 	metabolites = metabolite_config.metabolite_dict(model)
% 	reactions = reaction_config.reaction_dict(model, model_solution)
% 	objective = options_setup.get_objective(model)
%   # Get minimum/maximum fluxes for building the metabolic network
% 	flux = max(list(map(abs, (model_solution.fluxes))))
%   # Properly format the outputs
% 	assessment = meteor_functions.assessment_output(ATP_produced, ATP_metabolism,
%                 ATP_burned, ATP_biomass, ATP_waste, NADP_produced,
%                 NADP_metabolism, NADP_biomass, NADP_waste)
% 	result_categories = meteor_functions.category_dict(model, model_solution,
%                         atpwaste_reactions, ATP_waste, nadpwaste_reactions,
%                         NADP_waste, atpbiomass_reactions, atpburned_reactions,
%                         nadpbiomass_reactions)
% 	return metabolites, reactions, assessment, flux, model, result_categories, objective
% \end{lstlisting} 
% \end{spacing}


Test the text. Now with \texttt{cobra.flux\textunderscore analysis}!
\chapter{My Second Content Chapter}
\label{chapter3label}

% This just dumps some pseudolatin in so you can see some text in place.
\blindtext

% \lstset{frameround=fttt,language=Python,numbers=left,breaklines=true}

\chapter{General Conclusions}
\label{conclusionlabel}

% This just dumps some pseudolatin in so you can see some text in place.
\blindtext

\phantomsection
\addcontentsline{toc}{chapter}{Appendices}

% The \appendix command resets the chapter counter, and changes the chapter numbering scheme to capital letters.
%\chapter{Appendices}
\appendix
\chapter{An Appendix About Stuff}
\label{appendixlabel1}
(stuff)

\chapter{Another Appendix About Things}
\label{appendixlabel2}
(things)

\chapter{Colophon}
\label{appendixlabel3}
\textit{This is a description of the tools you used to make your thesis. It helps people make future documents, reminds you, and looks good.}

\textit{(example)} This document was set in the Times Roman typeface using \LaTeX\ and Bib\TeX , composed with a text editor. 
 % description of document, e.g. type faces, TeX used, TeXmaker, packages and things used for figures. Like a computational details section.
% e.g. http://tex.stackexchange.com/questions/63468/what-is-best-way-to-mention-that-a-document-has-been-typeset-with-tex#63503

% Side note:
%http://tex.stackexchange.com/questions/1319/showcase-of-beautiful-typography-done-in-tex-friends
 
% You could separate these out into different files if you have
%  particularly large appendices.

% Actually generates your bibliography. The fact that \include is 
% the last thing before this ensures that it is on a clear page.
\bibliography{example}

% All done. \o/
\end{document}
